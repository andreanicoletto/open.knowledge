\chapter{Una stampante difettosa e la nascita del software libero}

Abbiamo visto come Internet costituisca, con il suo essere capillarmente esteso e i suoi
molteplici paradigmi, il miglior strumento di diffusione delle informazioni. Analizzato in questo
modo però risulta solamente una struttura vuota, come un apparato circolatorio all’interno
del quale non scorre del sangue. Il sangue sono le informazioni che sono racchiuse nei
server, la quantità di informazioni che fluisce nella rete dipende dalle condizioni a cui la
collettività deve sottostare per poterle utilizzare. Queste condizioni sono studiate dal diritto
d’autore e sono raccolte e definite nelle licenze. È l’aumento della restrittività delle licenze
d’uso del software proprietario che ha spinto molti programmatori ad aderire alle idee di
Richard Stallman e della sua Free Software Foundation. La nascita del Free Software e
dell’Open Source è ciò che ha incentivato l’apertura della rete alla distribuzione di contenuti
liberi di qualsiasi tipo e non solo di codice.

Per parlare di software libero è prima necessario parlare però di software proprietario.
Agli albori dell’informatica la questione della condivisione del codice si poneva in modo
diverso da quello attuale: le macchine erano infatti spesso uniche e avevano determinate
specifiche tecniche molto diverse tra loro. Questo rendeva impossibile portare\footnote{Dall'inglese to port, adattare un software ad una piattaforma diversa da quella per cui è stato progettato.} un
programma da una macchina all’altra; è negli anni sessanta, con la nascita dei sistemi
operativi e con lo sviluppo dei primi linguaggi di programmazione di alto livello, ovvero quei
linguaggi più vicini al linguaggio umano che al linguaggio macchina, che ci si rende conto
dei vantaggi nel riutilizzare una stessa porzione di codice.

Uno degli OS più diffusi all’epoca era Unix, distribuito assieme all’hardware in
differenti versioni commerciali personalizzate dai produttori. Queste personalizzazioni
spesso facevano in modo che il software scritto per una distribuzione non funzionasse su
una distribuzione concorrente, impedendo di fatto la condivisione del software attraverso le
varie piattaforme e la pratica di patch\footnote{Dall’inglese to patch: rattoppare, aggiustare. In ambito informatico indica l’opera di modifica di un programma volta ad eliminare un
difetto o ad aggiungere una funzione. Con il sostantivo patch si intende il frammento di codice che costituisce la modifica.}. Le licenze diventarono rapidamente il mezzo con il
quale i produttori potevano riciclare codice obsoleto, rivendendolo apportando modifiche
minime.

Cominciò tutto con una stampante difettosa. È il 1980 e al MIT è appena arrivata una nuova
stampante, donata dalla Xerox per sostituire una macchina obsoleta. Le stampanti, si sa,
spesso si inceppano e questa non fa eccezione. Capita così che i ricercatori mandino in
stampa una serie di documenti e solo più tardi, quando uno di essi si reca a prendere i plichi
stampati, si accorgano che un foglio ha inceppato l’apparecchio impedendogli di trasferire
su carta i file, costringendo tutti a riprogrammare la coda di stampa e ad attendere altro
tempo per poter avere la copia fisica dei loro documenti. Il problema si presentava anche
sulla vecchia stampante, ma era stato arginato da Richard Stallman con una piccola modifica
al codice sorgente del driver. Questa modifica faceva sì che l’inceppamento
dell’apparecchio venisse notificato sui computer ad esso connessi, avvertendo
immediatamente tutti del problema e evitando uno spreco di tempo. Stallman tentò di
patchare anche la nuova stampante ma si rese conto di non poter accedere al suo codice
sorgente, poteva visualizzare solo un file scritto in binario che né lui né i suoi colleghi erano
in grado di comprendere e modificare. Recatosi per altri motivi alla Carnegie Mellon
University dove lavorava l’autore del codice, il Prof. Robert Sproull, Stallman tentò di
ottenerne una copia su cui lavorare. Il Prof. Sproull però, avendo firmato un accordo di non
divulgazione con la Xerox, si rifiutò di consegnare una copia del codice al ricercatore. Gli
NDA (Non-Disclosure Agreement) erano una pratica abbastanza nuova all’epoca, ma in
rapida diffusione.

Stallman percepì questa nuova tendenza come una minaccia alla possibilità di
modificare i programmi secondo il gusto personale o le necessità della comunità. Dal punto
di vista dell’industria informatica rappresentava invece una svolta nelle tattiche di mercato,
svolta che rendeva il software fonte di guadagno quanto, se non più, dell’hardware. La legge
etica non scritta di un’intera comunità era messa a rischio, etica basata, sin dalla sua nascita
una ventina di anni prima, sulla condivisione, sulla libertà e sulla convinzione che i contributi
individuali fossero un importante mezzo di sviluppo della comunità nel suo complesso.

Il rifiuto di condividere da parte di Sproull portò Stallman a interrogarsi sulla gravità
del problema e, in seguito, a fondare la Free Software Foundation \parencite{williams,stegman},
fondazione da cui nasceranno GNU\footnote{GNU (Acronimo ricorsivo significante GNU's Not Unix) è un sistema operativo unix-like, prodotto dalla Free Software Foundation, che si
distingue per essere stato il primo SO distribuito come software libero.} e la GPL\footnote{La GPL (General Public License) è la licenza con cui è distribuito GNU ed è nata con lo scopo di fornire agli sviluppatori di software
libero che garantisse le "Quattro Libertà" (cfr. capitolo \ref{chapter:4}) e potesse essere applicata senza modifiche.} e che saranno fonti di
ispirazione per enti come Creative Commons e la Wikimedia Foundation.