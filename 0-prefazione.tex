\chapter*{Prefazione}

Se in questo momento state leggendo questo scritto, composto con un editor di testo, è
perché nel corso degli ultimi cent'anni qualcuno si è impegnato ad elaborare gratuitamente
un piccolo pezzo di codice e lo ha reso pubblico. Questo frammento di codice è stato poi
aggiornato costantemente e raccolto assieme ad altri frammenti scritti da altri sviluppatori.
Questo processo, che nel precedente esempio è stato applicato al settore informatico, nasce
dalla volontà dei singoli individui di creare risorse e contenuti alla portata di chiunque, il cui
uso non sia dunque limitato dalla disponibilità di denaro, dall'appartenenza ad un
determinato gruppo di persone o da qualsiasi altro vincolo. Nulla ha mai influenzato la vita
umana su questo pianeta quanto la circolazione di idee e contenuti e questo libero moto non
è mai stato così rapido come negli ultimi cinquant'anni (Schneider et al. 2010), spinto
dall'avvento di Internet e dall'esponenziale crescita del suo numero di utenti (basti pensare
che nel 1969 ARPANET - la prima rete informatica – contava quattro computer collegati,
cresciuti a ventiquattro nel 1972 (Stewart 2000) e che l'anno scorso il numero stimato di
dispositivi collegati in rete era di circa diciassette miliardi (Lasse 2018).
La condivisione della conoscenza è una pratica intellettuale che possiamo ritrovare
nei contesti più disparati, dai database bioinformatici alle brevi tracce audio non protette da
copyright. Nelle prossime pagine cercherò di illustrare, analizzando la questione da diversi
punti di vista, come e perché questo movimento di condivisione si è sviluppato. In seguito,
mostrerò come questa ha cambiato in maniera radicale il modo in cui viviamo la nostra vita
e come, in epoche più recenti, ha rivoluzionato la diffusione dei risultati della ricerca
accademica.

Per poter indagare le ragioni storiche e sociali che hanno portato a questa rivoluzione,
dobbiamo però prima definire ciò che noi intendiamo come conoscenza o – più
specificatamente – individuare la definizione di conoscenza che più è congeniale al nostro
racconto storico. Potremmo sforzarci e scomodare i grandi pensatori, da Platone a John
Pollock, passando per Hume e Karl Popper, per individuare l’essenza ontologica della
conoscenza. Tuttavia, non ne abbiamo bisogno e ci limiteremo ad un approccio materialista.
Definiremo quindi la conoscenza come bene e come risorsa.