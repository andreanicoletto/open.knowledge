\chapter{Perché l’Open funziona e come ha cambiato la nostra vita}

L’Open nasce quando delle persone con interessi comuni condividono degli obiettivi e
formano una comunità basata sulla fiducia reciproca, con lo scopo di raggiungerli.
L’organizzazione di questa comunità non è di tipo gerarchico, è un’auto-organizzazione fra
pari dove modifiche agli obiettivi possono essere proposte da tutti, le task sono divise in
base alle capacità dei singoli membri ma chiunque può seguirne lo sviluppo suggerendo e
apportando migliorie. Requisito necessario per il mantenimento della coesione della
comunità e per la cooperazione è la fiducia reciproca: ogni singolo individuo della comunità
rispetta il lavoro degli altri e si aspetta che gli altri rispettino il suo, quando si presentano
delle decisioni non sono mai prese in modo arbitrario da gruppi ristretti di individui, ma sono
prese in seguito a discussioni in modo molto simile a come accadrebbe in una democrazia
diretta. È importante ricordare che l’Open si basa sulla gift economy – dunque sul valore
d’uso degli oggetti e delle azioni e non sul valore di scambio come l’economia di mercato –
e che i membri della comunità si aspettano che l’impegno e lo sforzo che mettono nel
progetto venga contraccambiato dagli altri membri. È una struttura più simile a quella di una
famiglia che a quella di un consorzio, gli individui creano relazioni con gli altri partecipanti in
seguito alle interazioni necessarie a portare avanti i processi produttivi e decisionali; queste
relazioni sono quelle che tengono la comunità unita. È Internet il mezzo grazie al quale l’Open
è possibile; senza la possibilità di comunicare e condividere in tempi rapidi attraverso lunghe
distanze sarebbe pressoché impossibile attivare un progetto Open di successo. La rete
inoltre permette agli individui fuori dalla comunità di conoscere, visionare, commentare ed
eventualmente promuovere il progetto, dando potenzialmente una spinta in termini di
notorietà che si può ripercuotere sul numero di membri della comunità e dunque sulla
velocità di sviluppo del progetto stesso \parencite{goldman}.

La filosofia Open ha cambiato le nostre vite, possiamo trovare prodotti dell’ingegno
diffusi in modo libero in moltissimi oggetti che utilizziamo quotidianamente, il vero motore
del progresso è la collaborazione e senza di essa il mondo non sarebbe quello che è \parencite{yeaton},
non esisterebbero aeroplani, vaccini, smartphone, infrastrutture energetiche e molti
altri strumenti su cui si basa la nostra vita.