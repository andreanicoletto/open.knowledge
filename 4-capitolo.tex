\chapter{Un bar di Ivrea e i nuovi mezzi di diffusione, miglioramento e avanzamento del sapere e della cultura}
\label{chapter:4}
Il movimento per il software libero è stato pioniere e, per certi aspetti, padre di tutta
una serie di iniziative volte non solo a rendere disponibile a tutti la conoscenza, ma anche a
cercare di mantenerne le evoluzioni. I concetti di Open Source e di software libero non
coincidono\footnote{La differenza consiste principalmente nel fatto che per il Software Libero le implicazioni etiche hanno un ruolo più importante di quelle
pratiche, mentre l'Open Source a volte non prende nemmeno in considerazione l'aspetto etico della condivisione dell'opera. La spaccatura
tra i due movimenti trova origine in diverse restrizioni di reciproche licenze: un software libero per esempio deve essere ridistribuito
forzatamente con la stessa licenza con il quale è stato prodotto, impedendo di fatto l'utilizzo di software libero in un prodotto commerciale;
dall'altro lato è considerato Open Source anche quel software che è visibile a tutti ma non può essere modificato e non garantisce dunque
le "Quattro Libertà". Più importante è però la questione della tivoization, ovvero quella pratica di creare un sistema che incorpora software
sotto una licenza di tipo copyleft in un hardware che non consente l'esecuzione di versioni modificate di tale software. Tale pratica è
accettata dalla comunità Open Source ma non dalla comunità per il Software Libero.} \parencite{stallman} e vanno ben oltre il semplice poter vedere il codice sorgente.
La Free Software Foundation promuove le cosiddette “Quattro Libertà”: utilizzo,
condivisione, studio e miglioramento (o modifica) del contenuto. Nel corso degli anni alcuni
individui si sono staccati dal movimento per il software libero criticando la “contagiosità”
della licenza GPL in quanto tutte le opere derivate da creazioni licenziate con essa devono
necessariamente mantenerla per tutte le successive iterazioni (impedendo dunque la
possibilità di utilizzare quelle opere, in particolare librerie software, come parti di altre opere).
Sorgono così centinaia di nuove licenze prive della clausola share-alike che permettono
dunque l’utilizzo del codice Open Source come parte di prodotti commerciali.

Con il passare del tempo la comunità digitale si è resa conto che il principio di
condivisione può essere applicato ad altre entità oltre che al software: nascono così licenze
applicabili a tutti i tipi di opere dell’ingegno come le Licenze Creative Commons. Queste si
prepongono lo scopo di coprire tutte le situazioni comprese fra il copyright completo e il
pubblico dominio, definendo le libertà concesse dall’autore e le condizioni di utilizzo
dell’opera. Contemporaneamente allo sviluppo di queste licenze sorgono organizzazioni il
cui scopo è promuovere e diffondere contenuti liberi diversi dal software come: immagini,
video, audio e testi, dando così origine all’Open Content.

Uno dei più famosi enti di questo tipo è la Wikimedia Foundation che gestisce progetti come
Wikipedia e Wikimedia Commons, portali che oggi figurano nella lista dei siti più consultati
al mondo e che sono in costante crescita: Wikimedia Commons ha quasi triplicato il numero
di file archiviati rispetto al 2012 (raggiungendo quasi i quaranta milioni) e Wikipedia ha
superato a dicembre dello scorso anno la soglia delle quarantanove milioni di voci
(mantenendo una crescita costante che oscilla attorno all’1\% mensile) \parencite{zachte}.

Oltre che di Open Source e Open Content si parla sempre più spesso di Open Science,
Open Data, Open Economics, Open Education e tanti altri termini Open tramite i quali si
promuove la condivisione delle informazioni. Un ottimo esempio di uno degli aspetti
dell’Open Science è Ensembl – ma anche GenBank – una banca dati bioinformatica che
raccoglie sequenze nucleotidiche prodotte dai laboratori di tutto il mondo. Un progetto del
genere sfrutta la rete per fornire ai ricercatori di tutto il mondo (gli utenti di Ensembl sono
per la maggior parte ricercatori ma chiunque può accedervi in modo libero e gratuito) dati
aggiornati su uno dei temi caldi delle scienze \parencite{ensembl}, permettendo alla ricerca di
progredire molto più velocemente di quanto non farebbe se solo un gruppo ristretto di
persone potesse avere accesso a questi dati.

Un altro interessante paradigma dell’Open è sicuramente l’Open Hardware del quale
uno dei primi e più importanti progetti è Arduino\parencite{arduino}. Ideato ad Ivrea da un
gruppo di ricercatori dell’Interaction Design Institute, prende il nome dal bar dove si
riunivano i fondatori del progetto. Arduino è una piattaforma hardware che ha rivoluzionato
il modo di apprendere le basi dell’elettronica e il modo di costruire prototipi \parencite{lahart}.
È costituito da una scheda elettronica dotata di un microcontrollore e diversi connettori per
l’Input/Output sia analogico che digitale che permettono di collegarla a un grande numero
di sensori e componenti, rendendola uno strumento estremamente versatile che può essere
applicato nei campi più svariati: dalla costruzione di un sismografo a quella di una stazione
meteo, ma può anche essere usato come “cervello” di una stampante 3D. Gli schemi
hardware e il software collegato sono liberi, dando a chiunque la possibilità di costruirsi una
scheda propria o di produrne un clone o una versione modificata compatibile, tutto questo
favorendo la concorrenzialità e portando vantaggi in termini di qualità e costi all’utente finale.