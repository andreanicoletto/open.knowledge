\chapter{La conoscenza come bene e come risorsa}

La Terra è un sistema in continuo mutamento, sia dal punto di vista ambientale (con il
costante variare delle condizioni geofisiche e climatiche del pianeta) sia dal punto di vista
sociale. La comunità umana è parte del ‘sistema Terra’ e lo influenza pesantemente, infatti i
mutamenti sociali influiscono sulle alterazioni ambientali e viceversa. Le continue
trasformazioni nelle società, come possono essere le guerre, le crisi economiche e le
rivoluzioni, modificano i comportamenti delle persone e il loro modo di vivere. Il ventesimo
secolo è stato quello che ha visto la nascita della globalizzazione – a partire dai due conflitti
mondiali – fino alla nascita dell'e-commerce. Dalla metà del secolo scorso il mondo sta
diventando sempre più ricco, il prodotto mondiale lordo (GWP) è infatti in costante crescita,
se si esclude una piccola decrescita fra 2007 e 2008, quando il mercato immobiliare
statunitense è collassato su sé stesso e ha trascinato il mondo nell'attuale crisi economica.

Se negli ultimi venticinque anni il GWP è quasi raddoppiato è perché sono cambiati i
modi di produrre la ricchezza. La conoscenza, infatti, ha fatto il suo ingresso nel panorama
economico mondiale, la conoscenza – da questo punto di vista – non ha mai influito così
tanto sui mercati, sulla nascita di oggetti nuovi (che sono nelle nostre tasche) e sulla vita in
generale. Le imprese più produttive e dinamiche sono le stesse che producono e plasmano
conoscenze, perché sono in grado di sfruttare al meglio le risorse a loro disposizione e
rendersi più competitive sul mercato globale \parencite{greco}. La conoscenza
diventa così – in quanto oggetto di scambio (per esempio brevetti e proprietà intellettuali) –
una merce e in quanto tale è soggetta ad analisi di carattere economico.

Una delle più importanti e famose analisi del concetto di merce e delle sue proprietà
è quella di Karl Marx, scritta nel primo capitolo del Capitale, dove vengono individuate due
caratteristiche fondamentali delle merci: il valore d’uso e il valore di scambio. Il valore d’uso
è ‘l’utilità di una cosa’, ovvero il valore raccolto in potenza di tutto ciò che si può e si potrebbe
fare con tale determinata cosa. Per valore di scambio s’intende, invece, il rapporto
quantitativo tramite il quale si può scambiare una merce con un’altra; banalmente, il fattore
che influenza di più il prezzo che questa merce ha sul mercato \parencite{marx}. La conoscenza,
date le sue infinite potenzialità, ha valore d’uso infinito. Essa infatti è ciò che ci permette di
produrre. Tutto ciò che è creato o sfruttato dall’uomo è prodotto della conoscenza: sin dalle
prime comunità nomadi di cacciatori e raccoglitori è il conoscere che ha permesso all’uomo
di svilupparsi ed evolversi. Chi sopravviveva era chi era in grado di accendere un fuoco, chi
era in grado di costruire una lancia per andare a cacciare, o chi sapeva distinguere le bacche
commestibili da quelle velenose. Con l’evolversi delle società le conoscenze diventano
background dei primi sistemi economici basati sul baratto: chi sapeva costruire una lancia
comincia a costruirne altre allo scopo di scambiarle con chi, nel frattempo, sapendo
raccogliere bacche le aveva raccolte allo scopo di scambiarle.

Nel corso dei secoli il mondo è cambiato, sia dal punto di vista dell’organizzazione
delle società, sia dal punto di vista economico. Abbiamo assistito alla comparsa delle valute,
alla nascita dei mercati internazionali e all’esplosione del mondo della finanza. Le società
sono ora globali, stratificate e diversificate. Il livello di complessità del ‘sistema Terra’ (inteso
nelle sue componenti sociali, economiche e ambientali) è in costante crescita e provare a
controllarne o prevederne le variabili è impossibile. Una delle conseguenze di questo
aumento di complessità è un generale squilibrio nella distribuzione di tutte le risorse \parencite{balboni},
dal cibo al petrolio, dall’acqua alla conoscenza. Nel corso della storia abbiamo visto
come civiltà differenti si siano sviluppate in modo diverso in luoghi diversi del pianeta. Un
esempio sono le civiltà precolombiane, ovvero tutte quelle civiltà che occupavano il
continente americano prima della scoperta dell’America, le quali – sorte indipendentemente
dalle culture europee – dimostravano nel sedicesimo secolo conoscenze tecnologiche non
superiori a quelle degli antichi egizi. Come sarebbe cambiato il corso della storia se queste
civiltà avessero avuto accesso, nel corso del tempo, a tutte le innovazioni scoperte nel
mondo? Cosa sarebbe successo se Eratostene di Cirene, Aristarco di Samo e Ipparco di
Nicea fossero riusciti a diffondere in maniera efficiente le loro scoperte astronomiche?
L’umanità avrebbe già raggiunto Marte? Non lo possiamo sapere, fatto sta che la
conoscenza è una di quelle poche risorse che se condivise non riducono il loro valore e le
loro potenzialità bensì le moltiplicano \parencite{pulselli}. Condividere la conoscenza fa bene a
tutti: a chi la riceve perché guadagna un nuovo strumento e a chi la condivide perché
guadagna un'occasione di confronto e la possibilità che ciò che ha prodotto venga elaborato
e migliorato da altri individui.