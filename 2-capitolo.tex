\chapter{La rete delle reti, un nuovo mondo e un nuovo Medioevo}

Nel corso degli ultimi cinquant’anni ci ha permesso di cambiare radicalmente il mondo in cui
viviamo. È diventato un mondo parallelo, reale quanto quello che possiamo toccare. La
maggior parte delle comunicazioni del globo dipendono da esso, che si tratti del messaggio
di buongiorno inviato con lo smartphone o della ricevuta di avvenuto pagamento di una
grossa transazione alla borsa di Londra. Internet, la rete delle reti, il web, nato
dall’interconnessione tra reti informatiche di natura ed estensione diversa, rappresenta oggi
il sistema nervoso delle società industrializzate e in via di sviluppo. Il numero dei dispositivi
connessi in rete oggi è di circa 17 miliardi, attorno al 10\% in più rispetto all’anno scorso e
quattro miliardi di volte il numero di dispositivi collegati nel 1969 \parencite{lasse}, anno di
nascita della prima rete (ARPANET). È cambiato anche il tipo dei dispositivi che sono
connessi. Se prima del 2007 c’erano quasi solo computer, l’avvento degli smartphone rende
la tecnologia a portata di mano e a portata di tutti. Con il calo di prezzo che consegue alla
legge di Moore\footnote{La Prima Legge di Moore è un enunciato tratto da un'affermazione empirica di Gordon Moore, cofondatore di Intel; essa afferma: "La
complessità di un microcircuito, misurata ad esempio tramite il numero di transistori per chip, raddoppia ogni 18 mesi (e quadruplica
quindi ogni 3 anni)".}, questi dispositivi hanno raggiunto un numero di persone molto grande. Al
momento i mercati dove la loro vendita è in crescita maggiore sono proprio quelli delle
economie emergenti, come India e Brasile, che si stanno buttando a capofitto in investimenti
di aggiornamento delle reti e nella promozione del loro uso da parte della popolazione. La
novità sono però i dispositivi IoT \parencite{lasse} (Internet of Things, l’Internet delle cose),
ovvero tutti quegli apparecchi che sono connessi a Internet ma non ne permettono un
accesso diretto da parte dell’utente e trovano applicazione in innumerevoli campi: domotica,
robotica, industria automobilistica, ingegneria biomedica, telemetria, agricoltura, zootecnica
e tanti altri. Sono evoluzioni di apparecchi già esistenti a cui è stato fornito un accesso alla
rete. Questo accesso permette loro di raccogliere e trasmettere dati e di essere comandati
a distanza. Internet è un mezzo potentissimo che ci permette di fare cose che all’inizio del
secolo scorso nessuno si sarebbe immaginato e ciò ha generato una forte dipendenza del
mondo reale dal mondo digitale. Se Internet e le reti si spegnessero improvvisamente
assisteremmo a un collasso della società umana così come la conosciamo, o almeno di
quella parte di società umana che organizza con la rete tutte le attività che le forniscono
sussistenza. Le borse valori non potrebbero più funzionare e l’economia subirebbe una
violenta e repentina regressione, per non parlare dell’economia digitale non legata alla
finanza: bancomat, PayPal e tutti i sistemi di pagamento non fisici. Il sistema di distribuzione
delle merci non avrebbe più un mezzo efficace per gestire richieste e spostamenti; milioni
di tonnellate di merci deperibili già prodotte e destinate ai mercati non li raggiungerebbero
mai, portando, da un lato, a un ingente spreco di materie prime e risorse e, dall’altro, ad una
potenziale mancanza di rifornimenti di beni che potrebbero essere di prima necessità.

Ormai Internet si è integrato talmente bene nelle nostre vite che non abbiamo considerato
l’eventualità di non poterlo più utilizzare, non ci siamo preparati un piano B. Internet però
funziona, per sua stessa natura la rete è progettata in modo che il malfunzionamento di un
nodo\footnote{Uno degli obiettivi della prima rete, ARPANET, era quello di poter continuare a funzionare anche nel caso in cui un attacco atomico
sovietico avesse fatto saltare uno dei suoi nodi.} non influisca sul suo funzionamento complessivo. Questa particolare caratteristica
rende il problema dello spegnimento di Internet poco probabile. Un problema
indubbiamente più reale e con il quale è più probabile avere a che fare è quello che ha
indotto gli esperti a parlare di Digital Dark Age \parencite{maci2012b,ghosh}, Medioevo digitale.
Questa è la dicitura che descrive una situazione davanti alla quale ci siamo già trovati e
davanti alla quale ci troveremo sempre più spesso in futuro: sarà difficile o impossibile
visualizzare documenti digitali datati per via della scomparsa dei dispositivi in grado di
leggere i supporti hardware su cui sono salvati e dei software in grado di decodificarne i
formati. Internet è la Biblioteca d’Alessandria dei giorni nostri, con il vantaggio di non essere
situata in un unico luogo fisico, un’enorme raccolta di contenuti a cui tutti possono accedere
e a cui tutti possono contribuire, sperando che almeno questa non prenda fuoco, un giorno.
Il rischio è che fra cinquant’anni nessuno possa essere più in grado di leggere i libri contenuti
in questa biblioteca, milioni di terabyte di informazioni born-digital (nate digitali) che
rischiano di scomparire, lasciando i posteri privi di eredità culturale. Maglie deboli della rete
sono anche quei portali che accentrano enormi quantità di dati in un luogo solo \parencite{rossi},
dando a quel nodo importanza e vulnerabilità allo stesso tempo. Soggetti particolarmente
suscettibili a questo rischio sono le pubbliche amministrazioni \parencite{maci2012a} e gli enti che
raccolgono dati e informazioni, come la Wikimedia Foundation e i grandi database di open
data. Sono gli stessi enti che spingono il mondo digitale verso la democratizzazione delle
conoscenze, contrapponendosi alla monopolizzazione e incentivando la collaborazione fra
persone – oltre i confini geopolitici – allo scopo di migliorare in continuazione ciò che è il
principale prodotto dell’attività umana: il frutto del pensiero, la conoscenza e le opere
creative.