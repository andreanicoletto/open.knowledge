\chapter{Un nuovo Prometeo ci ha restituito il fuoco?}

Abbiamo visto come la conoscenza possa essere considerata uno dei beni più preziosi di
cui l’uomo dispone. Nel corso della storia il problema della trasmissione e condivisione del
sapere si è presentato molte volte: a partire dall’antichità dove la trasmissione era orale.
Oggi le più importanti fucine di conoscenza sono indubbiamente università e centri di
ricerca, i quali sono gli enti preposti, nell’attuale organizzazione della società, a indagare e
sviluppare i saperi. Necessità per l’efficiente progredire della conoscenza è che chi si
impegna in questo arduo compito lavori in quanto comunità e non in quanto singolo individuo
o piccolo gruppo \parencite{kuhn}. Allo scopo di permettere questa coesione è
necessario un sistema che raccolga, verifichi e ridistribuisca il prodotto del lavoro di ciascun
membro o gruppo di membri della comunità alla comunità intera. Il sistema attuale prevede
che a espletare questo compito siano degli editori, i quali si prendono in carico il compito di
formattare il lavoro, di sottoporlo a un processo di peer review e infine – se questo rispetta
determinati requisiti – di distribuirlo all’interno di riviste.

Questo sistema, che per l’accesso ad una rivista o ad un articolo prevede il
pagamento di un abbonamento o comunque di un “importo di lettura”, è retaggio di un
passato predigitale quando – per poter distribuire queste conoscenze – era necessario
crearne fisicamente delle copie, le quali dovevano essere trasportate attraverso gli oceani e
i continenti. Un passato in cui replicazione e distribuzione dell’informazione avevano un
costo, un costo di carta, inchiostro e carburante che giustificava dunque in qualche modo
quelli che erano i prezzi di accesso alla rivista. Ma oggi, con l’avvento della rete e delle
tecnologie dell’informazione questo sistema fatica a giustificare i compensi richiesti al
lettore, i quali risultano spesso un ostacolo all’obbiettivo della comunità, che queste
conoscenze, prodotte da essa ad essa ritornino per essere ulteriormente sviluppate \parencite{taylor}.

Il problema si complica ulteriormente se consideriamo il sistema nella sua completa
interezza, esistono migliaia e migliaia di riviste, per accedere dunque a tutta l’informazione
disponibile su un determinato argomento è necessario pagare molteplici abbonamenti.
Consistente è anche il problema opposto, dove determinate case editrici che pubblicano
molte riviste non permettono l’abbonamento solo ad alcune ma vendono pacchetti che ne
includono molteplici, anche riguardanti temi molto distanti fra loro, in modo di aumentare le
vendite di quelle minori \parencite{jeon}. Questo insieme di meccanismi è spesso
un ostacolo al reperimento di informazioni complete da parte sia di ricercatori all’interno
della macchina accademica – soprattutto in paesi dove i fondi all’istruzione sono limitati – sia
da parte di quelli all’esterno, non in grado di sostenere i costi degli abbonamenti. Ciò,
unitamente agli enormi margini di profitto degli editori \parencite{lariviere} – che in alcuni
casi si attestano attorno al 40\% – ha spinto, all’inizio del decennio, alcuni individui a opporsi
a questo metodo di distribuzione della conoscenza. Esempio degno di nota è sicuramente
la storia di Alexandra Elbakyan, studentessa kazaca, fondatrice di Sci-Hub un portale online
dove è possibile reperire gran parte della letteratura scientifica aggirando i paywall delle
case editrici \parencite{himmelstein}. Annoverata nella “Nature’s 10, ten people who
mattered this year” del 2016, la Elbakyan è diventata il simbolo di un mondo della ricerca
che si ribella alle logiche del mercato editoriale e punta all’Open Knowledge. È dunque Sci-
Hub la panacea di tutti i mali? Probabilmente no, si è infatti visto – in seguito ai fatti di “The
Cost of Knowledge” – come sarebbe più economicamente efficiente e rapido per la
comunità, per raggiungere la libera circolazione delle informazioni, un primo passaggio al
gold open access (quello ottenuto tramite il pagamento di fee alle riviste) invece che al green
open access (quello ottenuto pubblicando direttamente in repositories a libero
accesso)\parencite{suber}, Sci-Hub si presenta pertanto non come soluzione del problema della
gestione della conoscenza bensì come il primo vero indicatore dell’esistenza dello stesso.
Più che Prometeo – che disobbedisce alla legge divina – Alexandra Elbakyan è Antigone –
che disobbedisce alla legge dell’uomo per sé stessa e per i cittadini di Tebe – impegnata a
sovvertire l’ordine costituito a scopo di migliorare le condizioni della comunità. Se il modello
‘piratesco’ non è attualmente né sostenibile né accettato, qual è dunque la soluzione che
permetta alla comunità di avere massima diffusione delle conoscenze prodotte e allo stesso
tempo permetta di mantenere in piedi la macchina distributiva – che si occupa anche della
revisione e selezione dei paper – in modo efficiente? L’attuale modello gold – o corporate –
open access prevede il pagamento da parte dell’autore dell’articolo di una quota di
pubblicazione per poter rendere disponibile liberamente la sua opera, questa quota può
essere molto variabile e generalmente dipende da un complesso sistema di indici che
descrivono la qualità della rivista, fra i quali l’impact factor – il numero medio di citazioni per
anno di tutti gli articoli pubblicati nei due anni precedenti in un determinato journal \parencite{vanNoorden}.
Questo sistema tende a risolvere il problema dell’accesso alla conoscenza,
introducendone però un altro: in questo modo il costo di pubblicazione ricade sul gruppo di
ricerca che può pertanto avere vincoli riguardo alla rivista sul quale pubblicare. Allo stesso
modo compaiono riviste predatorie che in cambio della fee garantiscono la pubblicazione di
qualsiasi lavoro, comportamento che permette a conoscenze non accuratamente verificate
di essere mescolate alle altre, generando un grosso rumore di fondo fra le pubblicazioni \parencite{beall2013, beall2012}.
È evidente dunque come il modello corporate che, per sua stessa natura
e sopravvivenza, necessita di un guadagno e introduce nel meccanismo problemi legati al
fenomeno della concorrenza non possa essere ancora a lungo sostenibile se l’obbiettivo
finale è quello di garantire libertà a lettori e autori. La soluzione potrebbe essere il passaggio
al cosiddetto diamond open access (DOA), un modello non-profit dove sono le stesse
istituzioni accademiche o enti super partes che si impegnano a sostenere i costi di editing,
garantendo possibilità di poter pubblicare e libero accesso alle pubblicazioni in quanto
pubblico servizio. Questo sistema consentirebbe di reinvestire ciò che oggi è speso per
alimentare il corporate publishing nel DOA permettendo il raggiungimento dei vantaggi
dell’Open Access senza aumentare drasticamente i costi sostenuti dalle istituzioni
accademiche \parencite{fuchs}.